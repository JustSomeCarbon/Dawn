\documentclass{article}

\usepackage{geometry}
\geometry{margin=1in}
\usepackage{listings}

\title{
	\vspace{2in}
	\textmd{\textbf{Solace Running Documentation}}
	\vspace{3in}
}
\author{No\'e Garcia}
\date{March, 2023}

\begin{document}

\maketitle
\newpage

\section{Introduction}

This document is a running documentation for the Solace language. This documentation
contains information regarding value types, and general syntax along with simple examples.

\section{Types}

Solace is a statically typed language. All values are also immutable. The following table
contains all available types in Solace:

\begin{center}
\begin{tabular}{|r|l|}
	\hline
	Solace Type & Description \\
	\hline
	\hline
	int & general integer \\
	float & general floating point number \\
	char & general character \\
	string & general string value \\
	bool & general boolean value \\
	:sym & symbol value \\
	func & function type \\
	\hline
\end{tabular}
\end{center}

\subsection{Integers and Floats}

Built in number values are represented within Solace as integers or floats. Much like other
languages, integers represent whole values, and floats represent floating point real values.
integer values can be defined in the following manner:

\begin{lstlisting}
module:Main

func main int()
{
	// Declare a to be an integer and b to be a float
	a int = 1;
	b float = 0.1;
	
	// values of the same type can be defined with the use of commas:
	c, d int = 2, 3;
	
	0;
}
\end{lstlisting}

There are a number of arithmetic and comparison operations that are available for both integers
and floats. Like other languages, number values are able to be added, subtracted, multiplied,
and divided. Numbers are also able to be compared between one another. The following table 
contains operators for number types in Solace:

\begin{center}
\begin{tabular}{|r|l|}
	\hline
	Operators & Description \\
	\hline
	\hline
	$+$ & addition \\
	$-$ & subtraction \\
	$*$ & multiplication \\
	$/$ & division \\
	\hline
	$=$ & value assignment \\
	$==$ & equal to \\
	$>$ & greater than \\
	$<$ & less than \\
	$>=$ & greater than or equal to \\
	$<=$ & less than or equal to \\
	$!=$ & not equal to \\
	\hline
\end{tabular}
\end{center}

Arithmetic and comparison operators are performed between two given values. The following
is an example of each operator usage:

\begin{lstlisting}
module:Main

fun main int ()
{
	// assume a and b are integers:
	a, b int = 5, 10;
	
	// arithmetic can be performed
	addition int = a + b;
	subtraction int = b - a;
	division int = a / b;
	multiplication int = a * b;
	
	// numbers can also be compared:
	a > b;  // false
	a < b;  // true
	a >= b; // false
	a <= b; // true
	a == b; // false
	a != b; // true
	
	// The above can also be done with float type variables, or between integers
	// and float variables.
	
	0;
}
\end{lstlisting}

\subsection{Strings and Characters}

Solace has two different types for handling raw text values: characters and strings. Characters
represent a single character value, while strings represent a longer collection of character
values in succession. Character variables and values can be compared between one another
with the equality ($==$) operator. Strings can also be compared to one another in a
similar way. Character values can be concatenated together to result in a new string value
containing both character values. Strings can be concatenated together to result in a new
string value containing both original strings.

\begin{lstlisting}
module:Main

func main int ()
{
	// define character and string variables
	a char = 'a';
	b char = 'b';
	c string = "hello,";
	d string = " world!";
	
	// comparisons between character values
	a == a; // true
	a == b; // false
	// comparisons between string values
	c == c; // true
	c == d; //false
	
	// concatenate values together
	ab string = a | b; // "ab"
	cd string = c | d; // "hello, world!"
	
	0;
}
\end{lstlisting}

\subsection{Boleans and Symbols}

Boolean values can represent two different states: true and false. boolean values are
able to be compared through the use of comparison operations. Symbols are a unique type
as they represent defined names rather than raw values. There are two predefined symbols:
$:ok$ and $:err$.

\begin{lstlisting}
module:Main

func main int ()
{
	0;
}
\end{lstlisting}

\subsection{functions}

things

\section{Complex Data Types}

things

\section{Controll Flow}

things

\end{document}