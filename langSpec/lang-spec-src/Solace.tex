\documentclass{article}

\usepackage{geometry}
\geometry{margin=1in}

\title{
	\vspace{2in}
	\textmd{\textbf{Solace Spec Document}}
	\vspace{3in}
}
\author{Noe Garcia}
\date{12 November, 2022}

\begin{document}

\maketitle
\newpage

\section{Solace Overview}
This document contains the specification lists and requirements for the Solace Language.
Solace is a statically typed object oriented programming language with a focus on quick
construction and easy to read and write syntax. Unlike languages such as Java, not all
structures are required to be objects. The following document defines syntax, grammar, and
lexical aspects of the langauge. As the definition of the language grows, this document
will be updated.
Solace is a small project designed and constructed by one person who is learning more
as they are building the language.


\section{Types}
This section documents the available types within Solace. Types are broken into Atomic,
Composite, and Domain-specific.

\subsection{Atomic Types}
\begin{tabular}{r|l}
	reserved words & description \\
	\hline
	\hline
	void & null return type \\
	null & base type representing no data type \\
	int & integer data type, defaults to int32 \\
	int32 & integer data type up to 4 bytes (possible) \\
	int64 & integer data type up to 8 bytes (possible) \\
	float & floating point number type, defaults to float 64 \\
	float32 & floating point number type up to 4 bytes (possible) \\
	float64 & floating point number type up to 8 bytes (possible) \\
	char & character type \\
	string & string type \\
	bool & bolean type \\
	:sym & symbolic type \\
	\hline
\end{tabular}

\subsection{Composite Types}
\begin{tabular}{r|l}
	type name & description \\
	\hline
	\hline
	arrays & designating an array object, contains same type collections. \\
	maps & contain key/value pairs of designated types \\
	class & designating a class object definition \\
	\hline
\end{tabular}

\subsection{Domain-Specific}
There is nothing to note here right now. As Solace if flushed out more information will be added.


\section{Lexical Rules}
Like in many languages, Solace has rules as to what constitutes legal syntax for things such as
variable declarations, class naming, and function and method naming.

Solace defines a number of reserved words that are used in defining types, function definitions,
class definitions, data structures, and much more. The reserved words chosen for this language
are meant to be similar to other languages, such as Python and C. A lot of those reserved words
are defined for type declarations shown above, or the legal operators for the language defined
below. The following are the reserved words for the language that do not fit in under the other
sections, however, are just as crucial to the language definition.

\subsection{Reserved Words}

\begin{tabular}{r|l}
Reserved word & description \\
\hline
\hline
main & used to declare the main function of the program \\
class & used to declare a class structure \\
\hline
\end{tabular}


\section{Operators}
The following showcase some of the legal oporators for Solace

\begin{tabular}{r|l}
Oporator character(s) & description \\
\hline
\hline
+ & addition operation \\
- & subtraction operation \\
* & multiplication operation \\
/ & division operation \\
\% & modulo operation \\
> & greater than comparison \\
< & less than comparison \\
>= & greater than or equal to comparison \\
<= & less than or equal to comparison \\
== & equal to comparison \\
!= & not equal to comparison \\
\hline
\end{tabular}

\section{Punctuation}
Things


\section{Syntax}
Things


\section{Summary}
Solace is a simple object oriented programming language built as a simple project language.
This language is meant to be simple to read and write programs with, circumventing the verbose
nature of traditional object oriented languages. Solace is written utilizing Yacc, Bison, and C.
The goal right now is getting the language up and off the ground by implementing the specifications
defined above. From there, the language will be evaluated and this document will be updated on
where the language will build from there.


\end{document}
