\documentclass{article}

\usepackage{geometry}
\geometry{margin=1in}
\usepackage{listings}

\title{
	\vspace{2in}
	\textmd{\textbf{Solace Language Specification Document}}
	\vspace{3in}
}
\author{No\'e Garcia}
\date{February, 2023}

\begin{document}

\maketitle
\newpage

\section{Solace Overview}
This document contains the specification lists and requirements for the Solace Language.
Solace is a statically typed functional programming language with a focus on quick
construction and easy to read and write syntax. The following document defines syntax,
grammar, and lexical aspects of the langauge. As the definition of the language grows
and changes, this document will be updated.

Solace is an implementation of a functional programming language,
while taking aspects from many different languages such as C and Rust.
The main focus with this project is introducing the functional paradigm in a easily
digestible manner so as to minimize confusion for users who are most comfortable with
the imparative paradigm. Different ideas for how to do this may be explored, and as
such, the specifications of the language will most likely change until a
stable foundation is established.

Solace is a small project designed and constructed by one person who is learning more
as they are building the language. As such take all implementation lightly,
as Solace is built and polished.


\section{Types}
This section documents the available types within Solace. Types are broken into Atomic,
Composite, and Domain-specific. Solace is going to function as a general programming language,
with no specific focus on any domain (as of yet).

\subsection{Atomic Types}
\begin{center}
\begin{tabular}{|r|l|}
	\hline
	reserved words & description \\
	\hline
	\hline
	int & integer data type, defaults to int32 \\
	float & floating point number type, defaults to float 64 \\
	char & character type \\
	string & string type \\
	bool & bolean type \\
	:sym & symbolic type \\
	func & function type \\
	\hline
\end{tabular}
\end{center}

\subsection{Composite Types}
\begin{center}
\begin{tabular}{|r|l|}
	\hline
	type name & description \\
	\hline
	\hline
	list/array & designating an array type, contains same type collections. \\
	tpl & basic typle type for simple groupings. \\
	struct & designating a structure object definition for more complex data objects. \\
	\hline
\end{tabular}
\end{center}

\subsection{Domain-Specific}
Solace is meant to be general purpose at this point, so nothin domain specific has been defined.


\section{Lexical Rules}
Like in many languages, Solace has rules as to what constitutes legal syntax for things such as
variable declarations, function definitions, and naming criteria.

Solace defines a number of reserved words that are used in defining types, function definitions,
and data structures. The reserved words chosen for Solace are designed to be similar to
other languages, such as C, while taking a different approach. A lot of the
reserved words are defined for type declarations shown above, or the legal operators for the 
language that are defined below. The following are the reserved words for the language
that do not fit in under the other sections.

\subsection{Reserved Words}

\begin{center}
\begin{tabular}{|r|l|}
\hline
Reserved word & description \\
\hline
\hline
main & used to declare the main function of the program \\
use & used to include packages/libraries \\
module & used to define the module the file belongs to \\
ret & keyword used to return value(s) from function \\
\hline
\end{tabular}
\end{center}


\section{Operators}
The following showcase some of the legal oporators for Solace

\begin{center}
\begin{tabular}{|r|l|}
\hline
Oporator character(s) & description \\
\hline
\hline
$+$ & addition operation \\
$-$ & subtraction operation \\
$*$ & multiplication operation \\
$/$ & division operation \\
\% & modulo operation \\
$>$ & greater than comparison \\
$<$ & less than comparison \\
$>=$ & greater than or equal to comparison \\
$<=$ & less than or equal to comparison \\
== & equal to comparison \\
!= & not equal to comparison \\
\hline
\end{tabular}
\end{center}


\section{Syntax}
This section outlines an overview on the basic syntax of the Solace language.
The main goal of the syntax is to remain simple but flexible enough to build
interesting programs. This includes the ability for defining variables and functions,
higher order functions, and establishing basic program workflow.
The following showcase a simple program that defines a fibonacci function and 
the main function of the program. The main function is necessary for the solace to
run, similar to C type languages.

Solace will take an approach to syntax similar to C like syntax. Solace is a statically typed
langauge, so variable types must be defined upon their declaration. Code blocks are defined
through the use of brackets. nested blocks can be defined within function definitions
for pattern matching functionality.


The following is a showcase of a simple program:
\begin{lstlisting}
module.Main

func main int ()
{
	
	out int = fibonacci(7); // returns 13
	
	// there is a ret and return keyword, but is not needed.
	// the last statement in a block will be returned.
	ret 0;
}

func fibonacci int (n int)
{
	// embedded blocks can be used when pattern matching
	// and grouping functionality:
	{ (n <= 1) -> 1; }
	fibonacci(n-1) + fibonacci(n-2);
}

func factorial int (n int)
{
	// when matching a single parameter, it is possible to
	// omit the parameter for just the matched values.
	{ (0) -> 1; }
	n * factorial(n-1);
}
\end{lstlisting}

In maintaining with the traditional functional language paradigm, all functions are
designed to be pure. There are no plans as of this time to completely force pure functions.

Higher order functions are also a fundamental aspect to Solace. The following showcase
how functions are used in the context of other functions.

\begin{lstlisting}
func main int ()
{
	// things
}
\end{lstlisting}

\section{Summary}
Solace is a simple functional programming language built as a toy project language.
This language is meant to be simplistic to read and write programs with. Solace is written 
utilizing Yacc, Bison, and C, with the clang compiler.
The goal right now is getting the language up and off the ground by implementing the
specifications defined above. From there, the language will be evaluated and this document
will be updated on where the language will build from there.


\end{document}
