\documentclass{article}

\usepackage{geometry}
\geometry{margin=1in}

\title{
	\vspace{2in}
	\textmd{\textbf{Solace Spec Document}}
	\vspace{3in}
}
\author{Noe Garcia}
\date{12 November, 2022}

\begin{document}

\maketitle
\newpage

\section{Solace Overview}
This document contains the specification lists and requirements for the Solace Language.
Solace is a statically typed object oriented programming language with a focus on quick
construction and easy to read and write syntax. Unlike languages such as Java, not all
structures are required to be objects. The following document defines syntax, grammar, and
lexical aspects of the langauge. As the definition of the language grows, this document
will be updated.
Solace is a small project designed and constructed by one person who is learning more
as they are building the language.


\section{Types}
This section documents the available types within Solace. Types are broken into Atomic,
Composite, and Domain-specific.

\subsection{Atomic Types}
\begin{tabular}{r|l}
	reserved words & description \\
	\hline
	\hline
	void & null return type \\
	null & base type representing no data type \\
	int & integer data type, defaults to int32 \\
	int32 & integer data type up to 4 bytes \\
	int64 & integer data type up to 8 bytes \\
	float & floating point number type, defaults to float 64 \\
	float32 & floating point number type up to 4 bytes \\
	float64 & floating point number type up to 8 bytes \\
	char & character type \\
	string & string type \\
	bool & bolean type \\
	: & symbolic type, user defined
\end{tabular}

\subsection{Composite Types}
\begin{tabular}{r|l}
	reserved words & description \\
	\hline
	\hline
	array & designating an array object \\
	class & designating a class object definition \\
	struct & designating a general structure object
\end{tabular}

\subsection{Domain-Specific}
There is nothing to note here right now. As Solace if flushed out more information will be added.


\section{Lexical Rules}
Things


\section{Reserved Words}
Things


\section{Operators}
Things

\section{Punctuation}
Things


\section{Syntax}
Things


\section{Summary}
Things


\end{document}
