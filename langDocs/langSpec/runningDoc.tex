\documentclass{article}

\usepackage{geometry}
\geometry{margin=1in}
\usepackage{listings}

\title{
	\vspace{2in}
	\textmd{\textbf{Solace Running Documentation}}
	\vspace{3in}
}
\author{No\'e Garcia}
\date{March, 2023}

\begin{document}

\maketitle
\newpage

\section{Introduction}

Solace is a statically typed functional programming language with a focus on quick
construction and easy to read and write syntax.
This document is a running documentation for the Solace language. This documentation
contains information regarding value types, and general syntax along with simple examples.

\section{Types}

Solace is a statically typed language. All values are also immutable. The following table
contains all available atomic types in Solace:

\begin{center}
\begin{tabular}{|r|l|}
	\hline
	Solace Type & Description \\
	\hline
	\hline
	int & general integer \\
	float & general floating point number \\
	char & general character \\
	string & general string value \\
	bool & general boolean value \\
	:sym & symbol value \\
	func & function type \\
	\hline
\end{tabular}
\end{center}

There are a small number of composite types as well as the standard types listed above.
The following table contains all available composite types in Solace:

\begin{center}
\begin{tabular}{|r|l|}
	\hline
	Solace Type & Description \\
	\hline
	tpl & tupple type utilized for grouping values of different types \\
	array & array types utilized for grouping values of the same type \\
	struct & structure types for more complex data and value groupings \\
	\hline
\end{tabular}
\end{center}

\subsection{Integers and Floats}

Built in number values are represented within Solace as integers or floats. Much like other
languages, integers represent whole values, and floats represent floating point real values.
integer values can be defined in the following manner:

\begin{lstlisting}
module:Main

func main int()
{
	// Declare a to be an integer and b to be a float
	a int = 1;
	b float = 0.1;
	
	// values of the same type can be defined with the use of commas:
	c, d int = 2, 3;
	
	0;
}
\end{lstlisting}

There are a number of arithmetic and comparison operations that are available for both integers
and floats. Like other languages, number values are able to be added, subtracted, multiplied,
and divided. Numbers are also able to be compared between one another. The following table 
contains operators for number types in Solace:

\begin{center}
\begin{tabular}{|r|l|}
	\hline
	Operators & Description \\
	\hline
	\hline
	$+$ & addition \\
	$-$ & subtraction \\
	$*$ & multiplication \\
	$/$ & division \\
	$\%$ & modulo \\
	\hline
	$=$ & value assignment \\
	$==$ & equal to \\
	$>$ & greater than \\
	$<$ & less than \\
	$>=$ & greater than or equal to \\
	$<=$ & less than or equal to \\
	$!=$ & not equal to \\
	\hline
\end{tabular}
\end{center}

There are a small number of special operators. 

The concatenation operation takes two values
on the left and right of the concatenation operator and returns a new single value with
the left and right value concatenated together. Only values of the same type may be
concatenated. The new value returned will be of the same type as the original values.
There is a single case in which the concatenate returns a different value than that of the
values given, and that is with the concatenation of characters. The concatenation of two
character values together will result in a string.

The ignore character operator functions exactly as its name implies. Values that are stored
using the ignore chracter operator will be forgotten. This operator is utilized when a value
is no longer necessary and can be safely forgotten.

The action delegation arrow operator is a main tool utilized in the control flow of any
Solace program.

\begin{center}
\begin{tabular}{|r|l|}
	\hline
	Special Operators & Description \\
	\hline
	\hline
	$|$ & concatenation of left and right \\
	\_ & value ignore character \\
	$->$ & action delegation \\
	\hline
\end{tabular}
\end{center}

Arithmetic and comparison operators are performed between two given values. The following
is an example of each operator usage:

\begin{lstlisting}
module:Main

fun main int ()
{
	// assume a and b are integers:
	a, b int = 5, 10;
	
	// arithmetic can be performed
	addition int = a + b;
	subtraction int = b - a;
	division int = a / b;
	multiplication int = a * b;
	
	// numbers can also be compared:
	a > b;  // false
	a < b;  // true
	a >= b; // false
	a <= b; // true
	a == b; // false
	a != b; // true
	
	// The above can also be done with float type variables, or between integers
	// and float variables.
	
	0;
}
\end{lstlisting}

\subsection{Strings and Characters}

Solace has two different types for handling raw text values: characters and strings. Characters
represent a single character value, while strings represent a longer collection of character
values in succession. Character variables and values can be compared between one another
with the equality ($==$) operator. Strings can also be compared to one another in a
similar way. Character values can be concatenated together to result in a new string value
containing both character values. Strings can be concatenated together to result in a new
string value containing both original strings.

\begin{lstlisting}
module:Main

func main int ()
{
	// define character and string variables
	a char = 'a';
	b char = 'b';
	c string = "hello,";
	d string = " world!";
	
	// comparisons between character values
	a == a; // true
	a == b; // false
	// comparisons between string values
	c == c; // true
	c == d; //false
	
	// concatenate values together
	ab string = a | b; // "ab"
	cd string = c | d; // "hello, world!"
	
	0;
}
\end{lstlisting}

\subsection{Boleans and Symbols}

Boolean values can represent two different states: true and false. boolean values are
able to be compared through the use of comparison operations. Symbols are a unique type
as they represent defined names rather than raw values. There are two predefined symbols:
$:ok$ and $:err$.

\begin{lstlisting}
module:Main

func main int ()
{
	// function call will return two values: a symbol and the original integer.
	// the following will return (:ok, 4)
	
	evenResult :sym, val int = isEven(4);
	
	// function call will return a single boolean value: true or false.
	// the following will return false
	
	oddResult = isOdd(6);
	
	0;
}

func isEven (:sym, int) (n int)
{
	{ (n%2 != 0) -> (:err, n);}
	(:ok, n);
}

func isOdd bool (n int)
{
	{ (n%2==0) -> false;}
	true;
}
\end{lstlisting}

Symbols are special values within Solace. Symbols aside from :ok and :err can be defined
for furhter use. Symbols hold special value within programs as their values are directly
tied to their name, and no other value. variables can contain symbol values.
The following is a simple example of symbol definition and use.

\begin{lstlisting}
module:Main

// define two new symbols
:sym {:apple, :orange}

func main int ()
{
	// define a symbol variable
	apple :sym = :apple;
	orange :sym = :orange;
	
	// symbols can be compared.
	:orange == :apple; // false
	:apple == :apple;  // true
	
	// symbol variables can also be compared
	apple == orange;   // false
	orange != apple;   // true
	
	0;
}
\end{lstlisting}

\subsection{functions}

Functions function like any other type in Solace. Unlike other types, functions are
defined using the func keyword before the name of the function, not after. Functions
can take multiple parameters but \textbf{must} return a sinle value. Functions cannot return
no value, or more than one value. To return multiple values, tuples may be used to
encapsulate all values to be returned.
Functions, like any other type, can be passed into other functions as parameters.
The following is an example of function definition and usage:

\begin{lstlisting}
module:Main

func main int ()
{
	// call a function and store the results
	result int = addTwo(2, 3);
	
	// call a function that returns another function and store it
	// wthin a variable
	func f = retFunc(); // prints: "Returning a function from a function"
	f();		    	// prints: "Function returned from another function"
	
	// define a function
	func double int (n int) -> n+n;
	
	// call the function that applies the passed function argument
	// on the given integer value
	doubleResult int = applyGivenFunc(double, 5); // 10
	
	0;
}

func addTwo int (a int, b int)
{
	a+b;
}

func retFunc func ()
{
	IO:out("Returning a function from a function");
	
	// return an un-named/anonymous function
	func () :sym -> {IO:out("Function returned from another function"); :ok;}
}

func applyGivenFunc int (f func, n int)
{
	f(n);
}
\end{lstlisting}

\section{Complex Data Types}

Among the basic types available in Solace, there are a small number of complex types. The inclusion
of complex types enable efficient grouping and defining of the more simple types of Solace. This
enables the construction of more interesting data values and interactions between information.

Solace offers the use of structure types. Similar to structures seen in languages such as C, structures
are defined using a special name and all included types in the structure as their signature. The 
following is a simple showcase of how to define and use a structure in a Solace program.

\begin{lstlisting}
module:Main

struct cake
{
	cakeType string;
	icingType string;
	candles int;
};

func main int ()
{
	// Define a new cake structure
	// The new chocolateCake structure instance is a cake with
	// the cakeType value as "chocolate", the icingType value as
	// "vanilla", and the candles value as 12
	chocolateCake cake = cake{"chocolate", "vanilla", 12};
	
	// Access the value fields of a structure instance
	chocolateCake:cakeType; // returns "chocolate"
	chocolateCake:candles;   // returns 12
	
	0;
}

\end{lstlisting}

\section{Variables and Naming}

Variables are essential to Solace, and are a core functionality of the language. Variables have a broad
naming convention, but there are a few restrictions on variable names.

Variable and function names cannot begin with a number. Names such as "$1value$" are not allowed; however,
numbers are viable anywhere else within a name, for example "$value2$" or "$here2there$" are viable 
names. Variable and function names can begin with and contain upper case letters. Names such as "$Dataval$"
and "$dataVal$" are both viable. Standard naming convention for variable and function names is camel case.
Names cannot start with or contain any special characters with the exception of being able to 
contain the underscore character: "\_". Names are able to utilize the underscore character within a name such
as "$some\_var$". The underscore character holds a special meaning where any value bound to the variable
whose name is soley the underscore character is dropped from memory.

\section{Controll Flow}

Solace defines its own control flow for program development. Rather than having if-else
statements, Solace utilizes pattern matching code blocks. Control flow is defined and
contained through the use of brackets, and the arrow operator. The following is an example
of simple control flow:

\begin{lstlisting}
module:Main

func main int ()
{
	// determine if an integer is even or odd
	evenOrOdd(5); // odd
	evenOrOdd(8); // even
	
	// function for a more complex control flow
	groupFunc(8); // Mod4
	groupFunc(9); // Mod2
	groupFunc(7); // NoMod
	
	// control flow with default functionality
	doubleIfEven(6); // 12
	doubleIfEven(5); // 5
	
	0;
}

// This function takes an integer and returns the string
// "even" if the integer is even and "odd" otherwise
func evenOrOdd string (n int)
{
	// pattern matching block
	// if n mod 2 is 0, do what is to right of arrow:
	// return even
	{(n%2 == 0) -> "even";}
	
	// otherwise return odd
	"odd";
}

// This function takes an integer and returns a string
// depending on if the integer is divisible by 4, 3, 2
// or none.
func groupFunc string (n int)
{
	// The control flow block. other patterns
	// begin on a new line with a bar-arrow character: |>
	{(n%4 == 0) -> "Mod4";
	|> (n%3 == 0) -> "Mod3";
	|> (n%2 == 0) -> "Mod2";}
	"NoMod";
}

func doubleIfEven int (n int)
{
	// The control flow block. The base case
	// is defined with no pattern, just an arrow
	{(n%2 == 0) -> n*2;
	|> -> n;}
}

\end{lstlisting}

\end{document}
