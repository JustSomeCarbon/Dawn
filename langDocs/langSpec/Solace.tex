\documentclass{article}

\usepackage{geometry}
\geometry{margin=1in}
\usepackage{listings}
\usepackage{hyperref}

% set the language for the examples
\lstset{language=Bash}

\title{
	\vspace{2in}
	\textmd{\textbf{The Solace Programming Language}}
	\vspace{3in}
}
\author{No\'e Garcia}
\date{February, 2023}

\begin{document}

\maketitle
\newpage

\section{Introduction to Solace}
Solace is a small functional programming language developed as a hobby project starting in
$2023$. Solace is built to be a bridge between imperative and  declarative functional programming
language domains. Built to be simple and easy to read and write with, Solace is a statically typed,
semi-pure functional language with basic support for simple tooling and functionality. While in
its infant stages, the language will grow to support more common aspects found in other
languages.

Functional programming can be intimidating for those unfamiliar to many who have never worked
with it before. For this reason, Solace is a case study for a programming langauge which aims to
make the introduction to the functional paradigm more familiar and easily approachable.

Solace is a general programming langauge as constructed, but domain specific library extensions
will be built to extend the capabilities of Solace into specific use cases. As it stands, there are
plans to epand on Solace by adding the ability to write logic centric programs.

This text is an introdution to the Solace language as it is, and should be used by anyone who is looking
to learn how to read and write in the language.


\section{Functional Programming}
What is functional programming? In essence, functional programming is the practice of building computer
programs using functions. While this sounds similar to other languages if you have worked with languages
such as C or Python, there are a few rules that the functions in functional programming languages follow
that are not followed in other imperative langauges.

The first is that functions should all accept at least one variable. And, each function may only have one
output. For every input of a function there is a designated output; No matter how many times the function is
run with that input, it will always result in the same output.

The Second is that functions do not access or manipulate variables or values outside of themselves. This means
that functions are not able to work with global variables in their computation. Only the variables passed into
the functions and variables defined within the function are able to be used in any computation within the
function. Building on top of this, functions do not create side effects. This means functions do not alter
variable values or states outside of its own scope. All functions are effectively stateless and perform the
exact same every time they are called with the same input (as mentioned in the above point).

Lastly, all data in functional programs are immutable. When a variable is declared, its value is not able to
be changed. This makes values much more safe to use in complex operations that may share the same values. Building
on top of this, functional programming languages do not support loops that are found in imperative domains, such as
$for$ and $while$. Because values are immutable, the condition value in a $while$ loop would never change, likewise with
the $for$ loops, the variable defined in its scope would never be able to increase or decrease in value, making the loop
impossible in the functional domain. Rather than using loops, however, recursion is used when looping is necessary
in some computation.

Functional programming is a shift in how programs should be written. This shift makes for code that is more safe and easy to
understand. In the following sections, this book will go into depth of the Solace programming language and provide
information on the data types and structures of the language as well as how the language is written, used, and understood.


\section{Domain Specific Language}
Solace is a general purpose functional programming language built without a specific domain in mind. There are a number of
general purpose programming languages that exist today, so there are a number of languages to choose from in that regard,
Solace aims to be a hobby/playground/research language. The functional domain, while gaining traction in use, is still
regarded as a fringe domain to build production level code with. There is plenty to be done in terms of functional language
development, and Solace is a testing ground for that.

Beyond functional programming, logic programming is another domain of programming development that has not been brought out
of the academic sphere much. Solace is the perfect language to begin to test aspects of logic programming being brought into a
toy language to play around with and to understand its respective use cases better.


\section{Getting Started}
Solace is written in C and compiled with the $Clang$ compiler. Apart from this, the lexer for Solace is written in $Flex$
(also known as fast lexical analysis generator), and the parser is written with GNU Bison. Solace is hosted on github
\href{https://github.com/JustSomeCarbon/Solace/tree/main}{here} and can be cloned to a local machine using git. After
Solace is installed, it can be built by typing the following into the console while within the $Solace/src$ project directory:

\begin{lstlisting}
 \$ make
\end{lstlisting}

This will run the make file in the source directory and build the language from source. Once this completes, the executable
$./solx$ file will be available. This is the Solace compiler program. To compile a Solace file run the following with the
Solace source file to compile:

\begin{lstlisting}
 \$ ./solx solaceFileToCompile.solc
\end{lstlisting}

All Solace source files should have the $.solc$ file extension to signal to the compiler that what its looking at is a
Solace source code file. The compiler is also able to handle Solace source code files that do not have file extionsions.
If a source code file named $Fibonacci$ (with no file extension) the compiler takes the file and adds the $.solc$ extension
to the end of a file.

\subsection{Writing your First Lines}
Things


\section{Types}
Types are a foundational building block to all programming languages. Different types are represented to the computer using
different sizes of bytes, and allow the computer to know what to expect when we define some value for it to store for later
use. In languages like C, variables are statically typed. This means that the author has to explicitly define what type a
variable is at its declaration. Solace is also statically typed. To define a variable in Solace, the name of the variable is
first given followed by the type of the variable, and then the assignment of the variable.

Types can be broken down into two different classifications for Solace: atomic and composite. Atomic types are the basic
building blocks of the typing system within the language, while composite types are types that are built on top of atomic
types, used to handle the structure and storage of data.

Solace has general atomic types that are present in many other programming languages: integers, floating point numbers, characters,
strings, booleans, functions, and symbols.

\subsection{Atomic Types}
\begin{center}
\begin{tabular}{|r|l|}
	\hline
	reserved words & description \\
	\hline
	\hline
	int & integer data type, defaults to int32 \\
	float & floating point number type, defaults to float 64 \\
	char & character type \\
	string & string type \\
	bool & bolean type \\
	:sym: & symbolic type \\
	func & function type \\
	\hline
\end{tabular}
\end{center}

Integers are used to handle and store whole numbers. Numbers like $0$ or $-14$ can be stored in a variable with the integer type. Float
types are used to handle real numbers. Floats can store and handle values such as $0.1$ or $0.0001$. Characters handle single character
values such as $'a'$, while strings can handles multiple characters strung together like $"hello, world!"$. Booleans handles values like
$true$ and $false$. Function types are used to store whole functions. In functional programming languages such as Solace, functions are
treated like any other type. Symbols are an interesting type, as they store literal values. A symbol value $:apple:$ is only equivalent
to itself, and represents its literal value as defined by the author. Symbols can be any string of characters (excluding special
characters and white space).

Composite functions, as was previously defined, are built on top of atomic types as a coalition enabling the structure and storage of
information. Solace provides the following composite types: list/arrays, tupples, and structures. These types are more complex than
atomic types. Arrays are able to store a collection of values within its definition, enabling storage of multiple values within a single
variable. Tupples are similar but more flexible than arrays. Arrays are a collection of values that are of the same type, tupples are able
to store multiple values of different types. Structures are the most flexible of the composite types, enabling the construction of a data
store capable of storing multiple variable values.

\subsection{Composite Types}
\begin{center}
\begin{tabular}{|r|l|}
	\hline
	type name & description \\
	\hline
	\hline
	list/array & designating an array type, contains same type collections. \\
	tupple & basic typle type for simple groupings. \\
	struct & designating a structure object definition for more complex data objects. \\
	\hline
\end{tabular}
\end{center}

\subsection{Defining and storing values}
Things


\section{Lexical Rules}
Like in many languages, Solace has rules as to what constitutes legal syntax for things such as
variable declarations, function definitions, and naming criteria.

Solace defines a number of reserved words that are used in defining types, function definitions,
and data structures. The reserved words chosen for Solace are designed to be similar to
other languages, such as C, while taking a different approach. A lot of the
reserved words are defined for type declarations shown above, or the legal operators for the 
language that are defined below. The following are the reserved words for the language
that do not fit in under the other sections.


\subsection{Reserved Words}

\begin{center}
\begin{tabular}{|r|l|}
\hline
Reserved word & description \\
\hline
\hline
main & used to declare the main function of the program \\
use & used to include packages/libraries \\
module & used to define the module the file belongs to \\
ret & keyword used to return value(s) from function \\
\hline
\end{tabular}
\end{center}


\section{Operators}
The following showcase some of the legal oporators for Solace

\begin{center}
\begin{tabular}{|r|l|}
\hline
Oporator character(s) & description \\
\hline
\hline
$+$ & addition operation \\
$-$ & subtraction operation \\
$*$ & multiplication operation \\
$/$ & division operation \\
\% & modulo operation \\
$>$ & greater than comparison \\
$<$ & less than comparison \\
$>=$ & greater than or equal to comparison \\
$<=$ & less than or equal to comparison \\
== & equal to comparison \\
!= & not equal to comparison \\
\hline
\end{tabular}
\end{center}


\section{Syntax}
This section outlines an overview on the basic syntax of the Solace language.
The main goal of the syntax is to remain simple but flexible enough to build
interesting programs. This includes the ability for defining variables and functions,
higher order functions, and establishing basic program workflow.
The following showcase a simple program that defines a fibonacci function and 
the main function of the program. The main function is necessary for the solace to
run, similar to C type languages.

Solace will take an approach to syntax similar to C like syntax. Solace is a statically typed
langauge, so variable types must be defined upon their declaration. Code blocks are defined
through the use of brackets. nested blocks can be defined within function definitions
for pattern matching functionality.


The following is a showcase of a simple program:
\begin{lstlisting}
module:Main

func main int ()
{
	
	out int = fibonacci(7); // returns 13
	
	// there is a ret and return keyword, but is not needed.
	// the last statement in a block will be returned.
	ret 0;
}

func fibonacci int (n int)
{
	// embedded blocks can be used when pattern matching
	// and grouping functionality:
	{ (n <= 1) -> 1; }
	fibonacci(n-1) + fibonacci(n-2);
}

func factorial int (n int)
{
	// when matching a single parameter, it is possible to
	// omit the parameter for just the matched values.
	{ (0) -> 1; }
	n * factorial(n-1);
}
\end{lstlisting}

In maintaining with the traditional functional language paradigm, all functions are
designed to be pure. There are no plans as of this time to completely force pure functions.

\section{Summary}
Solace is a simple functional programming language built as a toy project language.
This language is meant to be simplistic to read and write programs with. Solace is written 
utilizing Yacc, Bison, and C, with the clang compiler.
The goal right now is getting the language up and off the ground by implementing the
specifications defined above. From there, the language will be evaluated and this document
will be updated on where the language will build from there.


\end{document}
